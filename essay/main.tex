\documentclass[UTF8,a4paper,10 pt]{article}%字符标准,纸张,字体大小,文档类型,自行打印11pt,单位打印12pt
\usepackage{mathtools,amssymb,array,amsthm,amsmath}%数学宏包
\usepackage{geometry,ulem,graphicx,longtable,caption2,cite,fancyhdr,multicol,color}%通用宏包
\geometry{a4paper,left=2cm, right=2cm, top=2.6cm, bottom=3cm}%页面布局
\usepackage{ctex}
\usepackage{multirow} % Required for multirows
\newcommand\dd{\mathop{}\!\mathrm{d}}%定义微分算子
\newcommand\eu{\mathrm{e}}%定义自然常数
\newcommand\cbnt{\mathrm{C}}%定义自然常数\\
\newcommand\argm{\mathrm{A}}%定义自然常数
\renewcommand\bar{\overline}
\renewcommand\vec{\overrightarrow}
\usepackage{graphicx,graphics}
\usepackage{wrapfig}
\usepackage{xeCJK}

\usepackage{booktabs}
\usepackage[hidelinks]{hyperref}
\xeCJKsetup{
	CJKecglue={\:}
}
\AtBeginDocument{
	\let\mathbb\relax
	\DeclareMathAlphabet{\mathbb}{U}{msb}{m}{n}
}
\setlength{\lineskip}{8pt}
\setlength{\lineskiplimit}{8pt}




\usepackage{listings}
\usepackage{xcolor}
\lstset{
	basicstyle=\small\ttfamily,	% 基本样式
	keywordstyle=\color{blue!70!green}, % 关键词样式
	commentstyle=\color{yellow!30!green!70},   	% 注释样式
	stringstyle=\color{red!70!purple}, 	% 字符串样式
	backgroundcolor=\color{gray!5},     % 代码块背景颜色
	frame=leftline,						% 代码框形状
	framerule=18pt,%
	rulecolor=\color{purple!10!blue!10},      % 代码框颜色
	numbers=left,				% 左侧显示行号往左靠, 还可以为right ,或none,即不加行号
	numberstyle=\footnotesize,	% 行号的样式
	firstnumber=1,
	stepnumber=1,                  	% 若设置为2,则显示行号为1,3,5
	numbersep=7pt,               	% 行号与代码之间的间距
	aboveskip=.25em, 			% 代码块边框
	showspaces=false,               	% 显示添加特定下划线的空格
	showstringspaces=false,         	% 不显示代码字符串中间的空格标记
	keepspaces=true, 					
	showtabs=false,                 	% 在字符串中显示制表符
	tabsize=2,                     		% 默认缩进2个字符
	captionpos=n,                   	% 将标题位置设置为底部
	flexiblecolumns=true, 			%
	breaklines=true,                	% 设置自动断行
	breakatwhitespace=false,        	% 设置自动中断是否只发生在空格处
	breakautoindent=true,			%
	breakindent=1em, 			%
	title=\lstname,				%
	escapeinside=``,  			% 在``里显示中文
	xleftmargin=1.5em,  xrightmargin=0em,     % 设定listing左右的空白
	aboveskip=1ex, belowskip=1ex,
	framextopmargin=1pt, framexbottommargin=1pt,
    abovecaptionskip=-2pt,belowcaptionskip=3pt,
	% 设定中文冲突,断行,列模式,数学环境输入,listing数字的样式
	extendedchars=false, columns=flexible, mathescape=true,
	texcl=true,
	fontadjust
}%

\allowdisplaybreaks[4]






\begin{document}
	\setlength{\lineskip}{8pt}
	\setlength{\lineskiplimit}{8pt}
    
\begin{table*}[!ht]
    \renewcommand\arraystretch{1.5}
    \centering  
    \begin{tabular}{|p{3.5cm}<{\centering}|p{3.5cm}<{\centering}|}
        \hline
        {队伍编号}&MC2406053\\
        \hline
        题号&C\\
        \hline        
    \end{tabular}
\end{table*}

\setcounter{page}{1}
\pagenumbering{roman}

\noindent\rule{\linewidth}{1pt}
\begin{center}
    \Large 论文标题
\end{center}
\begin{center}
    \large\bf 摘要
\end{center}

摘要内容

\noindent{\bf 关键词:随机森林算法}


\newpage

\tableofcontents


\newpage
\setcounter{page}{1}
\pagenumbering{arabic}


\section{问题重述}
\subsection{问题背景}
在电商物流网络中,订单履约的过程包括多个环节,其中核心环节之一是分拣。分拣中心负责根据不同的目的地对包裹进行分类,并将它们发送到下一个目的地,最终交付给顾客。因此,提高分拣中心的管理效率对整个网络的订单履约效率和成本控制至关重要。

货量预测在电商物流网络中扮演着至关重要的角色。准确预测分拣中心的货物量是后续管理和决策的基础,有助于合理安排资源。通常,货量预测是根据历史货物量、物流网络配置等信息,来预测每个分拣中心每天和每小时的货物量。

分拣中心的货量预测与网络的运输线路密切相关。通过分析各线路的运输货物量,可以确定各分拣中心之间的网络连接关系。当线路关系发生变化时,可以根据调整信息来提高对各分拣中心货量的准确预测。

基于货量预测的人员排班是下一步需要解决的重要问题。分拣中心的人员包括正式员工和临时工两种类型。合理安排人员旨在完成工作的前提下尽可能降低人力成本。根据物流网络的情况,制定了人员安排的班次和小时人效指标。在确定人员安排时,优先考虑使用正式员工,必要时再增加临时工。

\subsection{问题描述}
\subsection{总体思路分析}

\section{符号说明与基本假设}
\subsection{符号说明}

\begin{table}[!ht]
\caption{符号说明}%标题
\centering%把表居中
\begin{tabular}{p{4cm}<{\centering}p{8cm}<{\centering}}%四个c代表该表一共四列,内容全部居中
\toprule%第一道横线
符号&说明 \\
\midrule%第二道横线 
&\\
\bottomrule%第三道横线

\end{tabular}
\end{table}

\subsection{基本假设}

\section{问题一的建模与求解}




\clearpage
\begin{center}
    \huge \bf 附录
\end{center}
\appendix
\section{附录1}

\begin{lstlisting}[language=python]
import matplotlib as mpl
import matplotlib.pyplot as plt
from matplotlib.axes import Axes
from mpl_toolkits.mplot3d import Axes3D
import numpy as np
from matplotlib.backends.backend_pdf import PdfPages
from matplotlib.animation import FuncAnimation
from scipy.integrate import odeint
from scipy import linalg as la
from scipy import optimize
import scipy

config = {
    "text.usetex": True,
    "text.latex.preamble": r"\usepackage{CJK}",  # 预先导入CJK宏包处理中文
}
plt.rcParams.update(config)
\end{lstlisting}

\end{document}